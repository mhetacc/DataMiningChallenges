%%
%% This is file `sample-manuscript.tex',
%% generated with the docstrip utility.
%%
%% The original source files were:
%%
%% samples.dtx  (with options: `all,proceedings,bibtex,manuscript')
%% 
%% IMPORTANT NOTICE:
%% 
%% For the copyright see the source file.
%% 
%% Any modified versions of this file must be renamed
%% with new filenames distinct from sample-manuscript.tex.
%% 
%% For distribution of the original source see the terms
%% for copying and modification in the file samples.dtx.
%% 
%% This generated file may be distributed as long as the
%% original source files, as listed above, are part of the
%% same distribution. (The sources need not necessarily be
%% in the same archive or directory.)
%%
%%
%% Commands for TeXCount
%TC:macro \cite [option:text,text]
%TC:macro \citep [option:text,text]
%TC:macro \citet [option:text,text]
%TC:envir table 0 1
%TC:envir table* 0 1
%TC:envir tabular [ignore] word
%TC:envir displaymath 0 word
%TC:envir math 0 word
%TC:envir comment 0 0
%%
%% The first command in your LaTeX source must be the \documentclass
%% command.
%%
%% For submission and review of your manuscript please change the
%% command to \documentclass[manuscript, screen, review]{acmart}.
%%
%% When submitting camera ready or to TAPS, please change the command
%% to \documentclass[sigconf]{acmart} or whichever template is required
%% for your publication.
%%
%%
\documentclass[manuscript,screen,review]{acmart}
%%
%% \BibTeX command to typeset BibTeX logo in the docs
\AtBeginDocument{%
  \providecommand\BibTeX{{%
    Bib\TeX}}}

%% Rights management information.  This information is sent to you
%% when you complete the rights form.  These commands have SAMPLE
%% values in them; it is your responsibility as an author to replace
%% the commands and values with those provided to you when you
%% complete the rights form.
\setcopyright{acmlicensed}
\copyrightyear{2025}
\acmYear{2025}
\acmDOI{XXXXXXX.XXXXXXX}
%% These commands are for a PROCEEDINGS abstract or paper.
%\acmConference[Conference acronym 'XX]{Make sure to enter the correct
%  conference title from your rights confirmation email}{June 03--05,
%  2018}{Woodstock, NY}
%%
%%  Uncomment \acmBooktitle if the title of the proceedings is different
%%  from ``Proceedings of ...''!
%%
\acmBooktitle{Data Mining: BeeViva Challenges,
  December 23, 2025, Padua, IT}
\acmISBN{978-1-4503-XXXX-X/2025/09}


%%
%% Submission ID.
%% Use this when submitting an article to a sponsored event. You'll
%% receive a unique submission ID from the organizers
%% of the event, and this ID should be used as the parameter to this command.
%%\acmSubmissionID{123-A56-BU3}

%%
%% For managing citations, it is recommended to use bibliography
%% files in BibTeX format.
%%
%% You can then either use BibTeX with the ACM-Reference-Format style,
%% or BibLaTeX with the acmnumeric or acmauthoryear sytles, that include
%% support for advanced citation of software artefact from the
%% biblatex-software package, also separately available on CTAN.
%%
%% Look at the sample-*-biblatex.tex files for templates showcasing
%% the biblatex styles.
%%

%%
%% The majority of ACM publications use numbered citations and
%% references.  The command \citestyle{authoryear} switches to the
%% "author year" style.
%%
%% If you are preparing content for an event
%% sponsored by ACM SIGGRAPH, you must use the "author year" style of
%% citations and references.
%% Uncommenting
%% the next command will enable that style.
%%\citestyle{acmauthoryear}

\input{lstlisting_settings}

\usepackage{fontawesome5}
\usepackage{subcaption}
\usepackage{tabularx}

\definecolor{changes}{HTML}{A42A04}

%%
%% end of the preamble, start of the body of the document source.
\begin{document}

%%
%% The "title" command has an optional parameter,
%% allowing the author to define a "short title" to be used in page headers.
\title{Data Mining: BeeViva Challenges}

%%
%% The "author" command and its associated commands are used to define
%% the authors and their affiliations.
%% Of note is the shared affiliation of the first two authors, and the
%% "authornote" and "authornotemark" commands
%% used to denote shared contribution to the research.
\author{Marco Bellò}
\email{marco.bello.3@studenti.unipd.it}

%\authornotemark[1]

\affiliation{%
  \institution{University of Padua}
  \city{Padua}
  \country{Italy}
}

%%
%% By default, the full list of authors will be used in the page
%% headers. Often, this list is too long, and will overlap
%% other information printed in the page headers. This command allows
%% the author to define a more concise list
%% of authors' names for this purpose.
\renewcommand{\shortauthors}{Bellò}

%%%%
%%%% The abstract is a short summary of the work to be presented in the
%%%% article.
\begin{abstract}
All R source code, images, \LaTeX\ sources and datasets (in csv format), can be found at the following repository: \url{https://github.com/mhetacc/DataMiningChallenges/}. 
\end{abstract}
%%
%%
%%%%
%%%% The code below is generated by the tool at http://dl.acm.org/ccs.cfm.
%%%% Please copy and paste the code instead of the example below.
%%%%
%%\begin{CCSXML}
%%<ccs2012>
%%   <concept>
%%       <concept_id>10010520.10010575.10011743</concept_id>
%%       <concept_desc>Computer systems organization~Fault-tolerant network topologies</concept_desc>
%%       <concept_significance>500</concept_significance>
%%       </concept>
%%   <concept>
%%       <concept_id>10010147.10011777.10011778</concept_id>
%%       <concept_desc>Computing methodologies~Concurrent algorithms</concept_desc>
%%       <concept_significance>500</concept_significance>
%%       </concept>
%%   <concept>
%%       <concept_id>10010147.10010919.10010172</concept_id>
%%       <concept_desc>Computing methodologies~Distributed algorithms</concept_desc>
%%       <concept_significance>500</concept_significance>
%%       </concept>
%%   <concept>
%%       <concept_id>10010147.10011777.10011014</concept_id>
%%       <concept_desc>Computing methodologies~Concurrent programming languages</concept_desc>
%%       <concept_significance>500</concept_significance>
%%       </concept>
%% </ccs2012>
%%\end{CCSXML}
%%
%%\ccsdesc[500]{Computer systems organization~Fault-tolerant network topologies}
%%\ccsdesc[500]{Computing methodologies~Concurrent algorithms}
%%\ccsdesc[500]{Computing methodologies~Distributed algorithms}
%%\ccsdesc[500]{Computing methodologies~Concurrent programming languages}
%%%%
%%%% Keywords. The author(s) should pick words that accurately describe
%%%% the work being presented. Separate the keywords with commas.
%%\keywords{Python, Raft, Shared Consensus, Gaming, Multiplayer, Multithreading, RPC, Pygame}
%%
%%\received{05/09/2025}
%%\received[revised]{18/09/2025}
%%\received[accepted]{24/19/2025}

%%
%% This command processes the author and affiliation and title
%% information and builds the first part of the formatted document.
\maketitle

\section{Rice Varieties}

\subsection{Preliminary Observations}

\subsubsection{Dataset}

All following considerations are made using the datasets provided on the challenge page (\textit{rice\_test.csv} and \textit{rice\_train.csv}), but it is worth noting that they both stem from an original one that can be found at the \href{https://www.muratkoklu.com/datasets/}{dataset source page}. I used it to compute an alternative version of the *test* dataset that contains the true attribute for \textit{Class}.

\begin{lstlisting}[language={Python},label={code:mergedatasets}, caption={Merge datasets}]
df1 = pd.read_csv("rice_test.csv").round(10)
df2 = pd.read_csv("Rice_Cammeo_Osmancik.csv").round(10)

keys = [col for col in df1.columns]
merged = df1.merge(df2, on=keys, how="inner")
\end{lstlisting}

\subsubsection{Scatter Plot}

\begin{figure}[h]
  \centering
  \includegraphics[width=\linewidth]{imgs/scatterplot_rice_loess.png}
  \caption{Scatter Plot for the \textit{rice\_train} dataset with LOESS smoothing lines for each class.}
    \Description{Scatterplot of a multivariate dataset.}
  \label{fig:scatterplot_rice_loess}
\end{figure}

From the \textit{scatter plot} we can infer that there are four features (\textit{Area}, \textit{Perimeter}, \textit{Convex\_Area} and \textit{Major\_Axis\_Length}) that seem to be extremely correlated

\subsubsection{Correlation Matrix}

We can use a correlation matrix to see exactly how related to each other the features are. The results are as follow.

\begin{table}
  \centering
  \caption{Correlation matrix of features}
  \label{tab:morphology_correlation}
  \begin{tabularx}{\textwidth}{lXXXXXXX}
    \toprule
    Area & Perimeter & Major\_Axis\_Length & Minor\_Axis\_Length & Eccentricity & Convex\_Area & Extent \\
    \midrule
    Area              &  1.00000000 &  0.96704011 &  0.90356325 &  0.79022701 &  0.34653177 &  0.99895272 & -0.06002223 \\
    Perimeter         &  0.96704011 &  1.00000000 &  0.97163180 &  0.63486482 &  0.53784650 &  0.97046788 & -0.12718015 \\
    Major\_Axis\_Length &  0.90356325 &  0.97163180 &  1.00000000 &  0.45675159 &  0.70625660 &  0.90390912 & -0.13562592 \\
    Minor\_Axis\_Length &  0.79022701 &  0.63486482 &  0.45675159 &  1.00000000 & -0.29393931 &  0.78975480 &  0.06192558 \\
    Eccentricity      &  0.34653177 &  0.53784650 &  0.70625660 & -0.29393931 &  1.00000000 &  0.34707340 & -0.19301157 \\
    Convex\_Area       &  0.99895272 &  0.97046788 &  0.90390912 &  0.78975480 &  0.34707340 &  1.00000000 & -0.06397213 \\
    Extent            & -0.06002223 & -0.12718015 & -0.13562592 &  0.06192558 & -0.19301157 & -0.06397213 &  1.00000000 \\
    \bottomrule
  \end{tabularx}
\end{table}

As suspected, all four features mentioned above have a high degree of correlation (over ninety percent). Once way to handle this is to either use models robust against collinearity, or to either combine or remove some of the correlated features.  

\subsubsection{Features' Importance}

To get a better idea on the importance of each feature, I trained a simple linear regression model, and looked at the result with \verb|summary(fit)|. The results follows.

\begin{table}
  \centering
  \caption{Regression coefficients and coefficients' importance}
  \label{tab:regression_coefficients}
  \begin{tabularx}{\textwidth}{lXXXXX}
    \toprule
    Parameter & Estimate & Std. Error & t value & Pr(>|t|) & Importance\\
    \midrule
    (Intercept)       & -2.152e+00 & 1.990e+00 & -1.081 & 0.279633 &  \\
    Area              &  5.601e-04 & 1.023e-04 &  5.474 & 4.79e-08 & *** \\
    Perimeter         &  8.440e-03 & 2.221e-03 &  3.800 & 0.000148 & *** \\
    Major\_Axis\_Length & -2.198e-02 & 5.709e-03 & -3.851 & 0.000120 & *** \\
    Minor\_Axis\_Length &  4.669e-02 & 1.213e-02 &  3.850 & 0.000121 & *** \\
    Eccentricity      &  4.534e+00 & 1.828e+00 &  2.481 & 0.013170 & * \\
    Convex\_Area       & -8.609e-04 & 9.418e-05 & -9.141 & < 2e-16 & *** \\
    Extent            &  7.146e-02 & 7.144e-02 &  1.000 & 0.317237 &  \\
    \bottomrule
  \end{tabularx}
\end{table}

The significance stars tells us visually that, with the exception of \textit{Minor\_Axis\_Length}, the features that are highly correlated to each other contributes strongly to the model, while the \textit{Eccentricity} and \textit{Extent} contribute very little.

We can also measure the total variance explained by all predictors combined using $R^2 = 0.6953849$, and we can check the standardized coefficients to better compare predictors' importance. The results follow.

\begin{table}
  \centering
  \caption{Standardized regression coefficients}
  \label{tab:standardized_coefficients}
  \begin{tabularx}{\textwidth}{lX}
    \toprule
    Parameter & Standardized Coefficient \\
    \midrule
    (Intercept)       & NA \\
    Area              & 1.95970669 \\
    Perimeter         & 0.60605955 \\
    Major\_Axis\_Length & -0.77272038 \\
    Minor\_Axis\_Length & 0.54256082 \\
    Eccentricity      & 0.18931314 \\
    Convex\_Area       & -3.09099064 \\
    Extent            & 0.01114328 \\
    \bottomrule
  \end{tabularx}
\end{table}

Once again, we can see that Extent and Eccentricity contribute very little to the overall model.

\subsection{Principal Components}

A good way to aggregate and simplify data is by decomposing it into its principal components (centered in zero and scaled to have unit variance). Results follow.

\begin{table}
  \centering
  \caption{Summary of principal components}
  \label{tab:pca_importance}
  \begin{tabularx}{\textwidth}{lXXXXXXXX}
    \toprule
    & PC1 & PC2 & PC3 & PC4 & PC5 & PC6 & PC7 & PC8 \\
    \midrule
    Standard deviation      & 2.2924  & 1.2436  & 0.9562  & 0.51393 & 0.10621 & 0.07758 & 0.04519 & 0.02052 \\
    Proportion of Variance  & 0.6569  & 0.1933  & 0.1143  & 0.03302 & 0.00141 & 0.00075 & 0.00026 & 0.00005 \\
    Cumulative Proportion   & 0.6569  & 0.8502  & 0.9645  & 0.99753 & 0.99894 & 0.99969 & 0.99995 & 1.00000 \\
    \bottomrule
  \end{tabularx}
\end{table}

We can see that more than ninety nine percent of overall variance can be explained with just three components, which means that not only we could make good prediction with lower degree data, but also that a 2D visualization of data that uses only two components should give us a good idea of the whole dataset.

\section{Phone Users}

\subsection{Preliminary Observations}

\subsubsection{Dataset}

Let's first briefly analyze the dataset.

\begin{itemize}
\small
  \item \textbf{For each user:}
  \begin{itemize}
    \item Plan type
    \item Payment method
    \item Sex
    \item Activation zone
    \item Activation channel
    \item Value-added service 1
    \item Value-added service 2
    \item \textbf{For each month:}
    \begin{itemize}
      \item $|\text{expensive calls}|$
      \item $|\text{cheap calls}|$
      \item $\text{time}(\text{expensive calls})$
      \item $\text{time}(\text{cheap calls})$
      \item $\text{cost}(\text{expensive calls})$
      \item $\text{cost}(\text{cheap calls})$
      \item $|\text{incoming calls}|$
      \item $\text{time}(\text{incoming calls})$
      \item $|\text{SMS}|$
      \item $|\text{calls to call center}|$
    \end{itemize}
  \end{itemize}
\end{itemize}

Considering that the target is monthly call time we can make the hypothesis that some features do not interest us, for example any monthly feature not concerning with call time itself, leaving us with only $time(expensive calls)$ and $time(cheap calls)$, which we could further assume can be combined since overall cost is not a concern. It goes without saying that all of the above will have to be be proven empirically.

\subsubsection{Scatter Plot}

I filtered out all not call time-related monthly data. The result (figure \ref{fig:phone_scatter_full}) is a bit hard to read due to its sheer size. Looking at it more up close we can infer some things: first of all we see that call data looks positively skewed (figure \ref{fig:phone_scatter_call}), and that there are some activation channels that see more call time than others, while activation regions seem to differ lightly between one another (figure \ref{fig:phone_scatter_geog+channel}). On the other hand, plan and sex seem to have an impact on call time (figure \ref{fig:phone_scatter_plan+sex}), while age does not seem to have a big impact, with the exception of the very young and very old (figure \ref{fig:phone_scatter_age}).

\begin{figure}[h]
  \centering
  \includegraphics[width=.8\linewidth]{imgs/scatterplot_phone_full_20perc.png}
  \caption{Scatter Plot for the filtered \textit{phone\_train} dataset.}
    \Description{Scatterplot of a multivariate dataset.}
  \label{fig:phone_scatter_full}
\end{figure}

\begin{figure}[h]
  \centering
  \includegraphics[width=.7\linewidth]{imgs/scatter_calldata.png}
  \caption{Scatter Plot for call time for \textit{phone\_train} dataset.}
    \Description{Scatterplot of multivariate call time.}
  \label{fig:phone_scatter_call}
\end{figure}

\begin{figure}[h]
    \centering
    \begin{subfigure}{0.35\textwidth}
        \includegraphics[width=\linewidth]{imgs/scatter_geog+channel_overcalltime.png}
        \caption{Activation zone (X-asis, L) and activation channel (X-axis, R) over call time (Y axis)}
        \label{fig:phone_scatter_geog+channel}
    \end{subfigure}
    \hspace{.005\linewidth}
    \begin{subfigure}{0.35\textwidth}
        \includegraphics[width=\linewidth]{imgs/scatter_plan+sex_over_calltime.png}
        \caption{Plan type (X-axis, L) and sex (X-axis, R) over call time (Y axis)}
        \label{fig:phone_scatter_plan+sex}
    \end{subfigure}
        \hspace{.005\linewidth}
    \begin{subfigure}{0.175\textwidth}
        \includegraphics[width=\linewidth]{imgs/scatter_age_over_calltime.png}
        \caption{Age (X-axis) over call time (Y-axis)}
        \label{fig:phone_scatter_age}
    \end{subfigure}

    \caption{Different features plotted over call time for the \textit{phone\_train} dataset.}
    \Description{Many multivariate graphs showing different features plotted over call time.}
    \label{fig:phone_scatter_allovercalls}
\end{figure}

\subsubsection{Call Features Over Everything Else}

Let's try to see if calls amount and time are related to other features.
First I plotted total calls taken and total calls time over nine months (figure \ref{fig:phone_calltimesandamount}). We can see that the graphs are almost identical from a trend standpoint, hinting at a strong correlation between number of calls and time spent calling. 

Then I wanted to see whether sex plays any role in total call time. Using the library \textit{dplyr} I computed all data in table \ref{tab:calltime_sex_cuts}. On average, men's calls seem to last 8\% longer. To mitigate the effect of outliers I filtered out the top 1\%, which still shows the same trend, albeit reduced to 5\%. Cutting even further the top 10\% shows a bigger increase in men's average call times of about 10\%. Lastly, I tired to log-transform the data. Even in this situation we see that men have longer calls, hinting that this predictor may be useful. 
I did exactly the same with payment methods, activation zones, activation channels and both value-added services, always cutting off the top 1\%,. All data can be seen in table \ref{tab:calltime_over_all}. There seem to be quite a bit of variance in call time compared to the tariff plan, which would make sense: some plans may be geared toward calling, while others are more suited sending SMS. We can see some variance on payment method too. Activation zone and activation channel seem to have some influence on call times, with the exception of zone eight, whose call times are way lower. On the other hand, customers that have activated either first or second added-value services seem to do longer calls.

\begin{figure}[h]
    \centering
    \begin{subfigure}{0.4\textwidth}
        \includegraphics[width=\linewidth]{imgs/monthly_calls.png}
        \caption{Total amount of calls per month}
    \end{subfigure}
    \hspace{.05\linewidth}
    \begin{subfigure}{0.4\textwidth}
        \includegraphics[width=\linewidth]{imgs/monthly_calltime.png}
        \caption{Total call time per month}
    \end{subfigure}

    \caption{Total calls and total call time per month for the \textit{phone\_train} dataset.}
    \Description{Graphs showing total calls and total call time per month.}
    \label{fig:phone_calltimesandamount}
\end{figure}

\begin{table}
  \centering
  \caption{Call time statistics by sex under different data cuts and transformations}
  \label{tab:calltime_sex_cuts}
  \begin{tabularx}{.75\textwidth}{lcccc}
    \toprule
    Type of cut & Sex & Number of customers & Total call time & Average call time \\
    \midrule
    None (raw)      & B & 5266 & 557266412 & 105823 \\
    None (raw)      & F & 1030 &  62493290 &  60673 \\
    None (raw)      & M & 3704 & 242226228 &  65396 \\
    \midrule
    Top 1\% cut     & B & 5201 & 498846876 &  95914 \\
    Top 1\% cut     & F & 1025 &  57203392 &  55808 \\
    Top 1\% cut     & M & 3674 & 215801665 &  58738 \\
    \midrule
    Top 10\% cut    & B & 4595 & 276063673 &  60079 \\
    Top 10\% cut    & F &  960 &  32400501 &  33751 \\
    Top 10\% cut    & M & 3445 & 130747836 &  37953 \\
    \midrule
    Log-transformed & B & 5266 & 557266412 & 10.00 \\
    Log-transformed & F & 1030 &  62493290 &  8.24 \\
    Log-transformed & M & 3704 & 242226228 &  8.58 \\
    \bottomrule
  \end{tabularx}
\end{table}

\begin{table}
  \centering
  \caption{Call time over different categorical features}
  \label{tab:calltime_over_all}
  \begin{tabularx}{.73\textwidth}{lccc}
    \toprule
    Feature & Number of customers & Total call time & Average call time \\
    \midrule
    Tariff plan 3 &  781 &  44811041  &  57376 \\
    Tariff plan 4 &   83 &  10575009  & 127410 \\
    Tariff plan 6 &  724 &  92214268  & 127368 \\
    Tariff plan 7 & 3564 & 489635465  & 137384 \\
    Tariff plan 8 & 4748 & 134616150  &  28352 \\
    \midrule
    Activation zone1 & 3507 & 288670845 & 82313 \\
    Activation zone2 & 3130 & 213438509 & 68191 \\
    Activation zone3 & 2342 & 197941555 & 84518 \\
    Activation zone4 &  921 &  71801024 & 77960 \\
    \midrule
    Activation channel 2 &  130 &  10643060 &  81870 \\
    Activation channel 3 &  408 &  36663577 &  89862 \\
    Activation channel 4 &   31 &   2369822 &  76446 \\
    Activation channel 5 & 7135 & 623961079 &  87451 \\
    Activation channel 6 &   47 &   4352853 &  92614 \\
    Activation channel 7 &  652 &  62777153 &  96284 \\
    Activation channel 8 &  111 &  12074180 & 108776 \\
    Activation channel 9 & 1386 &  19010209 &  13716 \\
    \midrule
    Value-added one: No & 7450 & 526620961 &  70687 \\
    Value-added one: Yes & 2450 & 245230972 & 100094 \\
    \midrule
    Value-added two: No & 9279 & 695404226 &  74944 \\
    Value-added two: Yes &  621 &  76447707 & 123104 \\
    \bottomrule
  \end{tabularx}
\end{table}

\subsubsection{Skewness}

Call time data is usually positively skewed, with many users having low usage and a small number of user with very high usage (e.g.,50.000 seconds). This is easily verifiable at a glance if we look at the density graph in figure \ref{fig:phone_call_density}. This effect can be mitigated by log-transforming the data, as shown in figure \ref{fig:phone_log_call_density}. Moreover, with the \textit{e1071} library I computed a skewness matrix for all features (appendix \ref{app:phone_skewness}): we can see that all monthly-based data (calls, call times, SMS, costs, etc) are highly positively skewed. To manage this problem we can either transform them (log, square root or Box-Cox transform), or use more robust models (such as random forest, Gamma/Poisson or quantile regression). Methods like linear. ridge or lasso regression, and even some non-parametric ones like KNN are not ideal. 

\begin{figure}[h]
    \centering
    \begin{subfigure}{0.4\textwidth}
        \includegraphics[width=\linewidth]{imgs/phone_y_hist.png}
        \caption{Call time density distribution}
        \label{fig:phone_call_density}
    \end{subfigure}
        \hspace{.05\linewidth}
    \begin{subfigure}{0.4\textwidth}
        \includegraphics[width=\linewidth]{imgs/phone_log_y_hist.png}
        \caption{Call time density distribution (log-transformed)}
        \label{fig:phone_log_call_density}
    \end{subfigure}
    \caption{Call time density distributions for the \textit{phone\_train} dataset.}
    \Description{Graphs showing call times density.}
    \label{fig:phone_calltimesandamount}
\end{figure}

\subsubsection{Correlation Matrix}

Since \verb|cor| function only works on numerical data, I transformed all categorical with the \verb|model.matrix| function. We can infer the following: age is correlated to all call-related monthly features by roughly 25\%, first and second value-added services are correlated to some, not all, monthly features by roughly 10\%, and the only features that share significant correlation are the monthly call-related ones, for example in month four, outgoing call time and call expenses share a correlation of 97\%, while for the seventh month the correlation drop to 57\%, which is still significant. 

\subsubsection{Linear Regression Insights}

A preliminary analysis can be done by plotting a linear regression fit (figure \ref{fig:phone_4lnplots}). We can see that the data presents high heteroskedasticity, skewness and in general it is not normally distributed. There are also potential outliers and influential points that could distort the model, such as point 9853. As for features relevance (appendix \ref{app:phone_ln_features_relevance}), for the most part, relevant features are the one related to monthly calls. It is interesting to notice that the second month does not seem to count much, even thought it presents a similar amount of call data compared to the others.

Log-transforming the target variable yielded better results (figure \ref{fig:phone_4loglnplots}), but the above mentioned problems did not disappear. Feature relevance held some surprises (appendix \ref{app:phone_log_ln_features_relevance}), for example categorical features contributed much more to the model and only a small subset of monthly call data was relevant.

\begin{figure}[h]
    \centering
    \begin{subfigure}{0.4\textwidth}
        \includegraphics[width=\linewidth]{imgs/4plots_ln.png}
        \caption{Linear regression fit plot}
        \label{fig:phone_4lnplots}
    \end{subfigure}
        \hspace{.05\linewidth}
    \begin{subfigure}{0.4\textwidth}
        \includegraphics[width=\linewidth]{imgs/4plots_ln_log.png}
        \caption{Linear regression fit plot (log-transformed)}
        \label{fig:phone_4loglnplots}
    \end{subfigure}

    \caption{Linear regression fit plots for \textit{phone\_train} dataset.}
    \Description{Graphs showing linear regression fits}
\end{figure}

\subsubsection{Principal Components}

Data as-is can be transformed into 101 principal components, and the cumulative proportion with only two is approximately 0.45. To get to 99\% of explained variance we need 69 components. Plotting the first two components shows us, once again, signs of heteroskedasticity and non-normal data distribution, as shown in figure \ref{fig:phone_pca}

\begin{figure}[h]
  \centering
  \includegraphics[width=.7\linewidth]{imgs/PCAphone.png}
  \caption{Scatter Plot for call time for the \textit{phone\_train} dataset.}
    \Description{Scatterplot of multivariate call time.}
  \label{fig:phone_pca}
\end{figure}

\subsubsection{Premilinary Observations Conclusions}

Data shows high skewness and heteroskedasticity, and some features do not seem to contribute much to the model. We probably should try to filter out some data, for example SMS amount, transform monthly call data, for example with log or Box-Cox transforms, and use some robust regression methods, such as quantile or tree-based approaches.

\subsection{Modeling}

\subsubsection{Lasso Regression}

I used lasso regression with cross-validation as a preliminary diagnostic tool to check which features drop or transform.
First, I made a baseline prediction without transforming any value, with the following mean square error: $MSE = 18091358$. Not great, but expected since we saw in the preliminary observations many signs of non-linearity.

I then tried log-transforming the target value (with all features), which resulted in a massive improvement: $MSE = 6.200236$. 

Then, on the not transformed target, I tried different predictions with different data filtering, with the following results: filtering out incoming calls, SMS and calls to the call center yielded a 3\% improvement over the all-features model ($MSE = 18040253$). Putting calls to the call center back in worsened the model slightly ($MSE = 18080968$). I thus decided to leave these features filtered out permanently.

Then, I tired to filter out out either both or any between calls value and calls amount, which produced a worse performing model in all instances. Then, I filtered out non monthly features, and discovered in the end that the best result ($MSE = 18019454$) can be obtained with the following features \textbf{filtered out}: payment method, value-added service two, activation zone and activation channel (plus, of course, the previously mentioned incoming calls, SMS and calls to the call center).

I then tried to pre-process the monthly features, such as adding together peak and off-peak values or predict on the average monthly call times. All yielded worse results.

Lastly, wit the best performing filtering, I log-transformed only the target ($MSE = 6.253648$), only the monthly features ($MSE =20415157$), and then both, which ultimately yielded the bet result: $MSE = 4.289544$. It is worth nothing that log-transforming the target on the filtered dataset produce a worse performing fit than a model fitted on all data and log-transformed target. 

To summarize: the best performing model is obtained by filtering out payment method, value-added service two, activation zone, activation channel, incoming calls, SMS and calls to the call center, and by log-transforming the target and the remaining monthly call-related features.

The complete list of filtering and transformations can be seen in appendix \ref{app:phone_lasso_filtering}

\subsubsection{Random Forest}

Considering all of the above a model inherently robust to non-normally distributed data should yield better results. Random Forest is a good example of such a model. I still log-transforming both target and predictor since it showed good results, and fitting this type of model requires a lot of time. I run the model with the \verb|ranger| method on a \verb|tuneLength| of ten with cross-validation. Compared to the previous challenge I had to give up on LOOCV due to memory constraints.
The resulting best fit (figure \ref{fig:phone_randomorest},with $mtry=8$, variance as the split rule and minimum node size of five) yielded the best mean square error overall: $MSE = 3.947784$, which is why I ultimately used this model to predict the target value.

\begin{figure}[h]
  \centering
  \includegraphics[width=.7\linewidth]{imgs/rf_plot.png}
  \caption{RMSE for random forest for the \textit{phone\_train} dataset.}
    \Description{RMSE for a random forest fit.}
  \label{fig:phone_randomforest}
\end{figure}

\subsection{Phone Users Conclusions}

The dataset was, in fact, highly not normally distributed and non-linear, so transforming the features greatly improved a linear fit such as lasso. Moreover, using a robust model yielded even better results, proving our hypothesis.

%%
%% The acknowledgments section is defined using the "acks" environment
%% (and NOT an unnumbered section). This ensures the proper
%% identification of the section in the article metadata, and the
%% consistent spelling of the heading.
%% \begin{acks}
%% To Robert, for the bagels and explaining CMYK and color spaces.
%% \end{acks}

%%
%% The next two lines define the bibliography style to be used, and
%% the bibliography file.
%\bibliographystyle{ACM-Reference-Format}
%\bibliography{report}
\newpage
\appendix
\section{Appendices}
\subsection{Phone call time skewness}

\begin{lstlisting}[language={bash},label={app:phone_skewness}, caption={Skweness matrix}]
        tariff.plan  activation.channel     q01.out.ch.peak    q01.out.dur.peak    q01.out.val.peak 
          -2.026605            1.073333            4.930308            5.883900           10.158610 
q01.out.dur.offpeak q01.out.val.offpeak       q01.in.ch.tot      q01.in.dur.tot          q01.ch.sms 
           9.336163           31.977903            4.322467            3.615376           37.220847 
    q02.out.ch.peak    q02.out.dur.peak    q02.out.val.peak  q02.out.ch.offpeak q02.out.dur.offpeak 
           4.011558            5.026925            6.120357            9.636244            9.202850 
      q02.in.ch.tot      q02.in.dur.tot          q02.ch.sms           q02.ch.cc     q03.out.ch.peak 
           3.657302            3.444048           38.464828           10.697464            3.742071 
   q03.out.val.peak  q03.out.ch.offpeak q03.out.dur.offpeak q03.out.val.offpeak       q03.in.ch.tot 
           4.144170            8.709344            9.948054           25.074765            3.359255 
         q03.ch.sms           q03.ch.cc     q04.out.ch.peak    q04.out.dur.peak    q04.out.val.peak 
          35.231226           12.146903            3.726583            4.415209            3.988589 
q04.out.dur.offpeak q04.out.val.offpeak       q04.in.ch.tot      q04.in.dur.tot          q04.ch.sms 
          12.366396           10.517232            3.362859            3.660378           33.513288 
    q05.out.ch.peak    q05.out.dur.peak    q05.out.val.peak  q05.out.ch.offpeak q05.out.dur.offpeak 
           3.578910            4.056863            4.242749           12.129119           12.598793 
      q05.in.ch.tot      q05.in.dur.tot          q05.ch.sms           q05.ch.cc     q06.out.ch.peak 
           3.450009            3.165951           38.804264            9.085293            3.364608 
   q06.out.val.peak  q06.out.ch.offpeak q06.out.dur.offpeak q06.out.val.offpeak       q06.in.ch.tot 
           4.037063           15.035554           12.511158           10.149751            3.386507 
         q06.ch.sms           q06.ch.cc     q07.out.ch.peak    q07.out.dur.peak    q07.out.val.peak 
          28.668984           13.747535            3.157588            4.088846            3.830199 
q07.out.dur.offpeak q07.out.val.offpeak       q07.in.ch.tot      q07.in.dur.tot          q07.ch.sms 
          12.907377           10.128761            3.237636            3.192192           28.628610 
    q08.out.ch.peak    q08.out.dur.peak    q08.out.val.peak  q08.out.ch.offpeak q08.out.dur.offpeak 
           3.963771            4.131527            4.117853            9.256204           11.536043 
      q08.in.ch.tot      q08.in.dur.tot          q08.ch.sms           q08.ch.cc     q09.out.ch.peak 
           3.972012            3.953185           20.317293            8.946827            2.927636 
   q09.out.val.peak  q09.out.ch.offpeak q09.out.dur.offpeak q09.out.val.offpeak       q09.in.ch.tot 
           3.646469           10.344301           15.974234           11.674903            2.728376 
         q09.ch.sms           q09.ch.cc                   y 
          16.048910           13.034136           10.820427 
\end{lstlisting}

\end{document}
\endinput
%%
%% End of file `sample-manuscript.tex'.
